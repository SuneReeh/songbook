\makeatletter

\DeclareRobustCommand{\@basecount}[3]{% Syntax: \@basecount{alphabet}{alphabet length}{counter}
  \ifnum #3 = 0%
    \csname @#1\endcsname{0}%
  \else%
    \edef\@baseResult{\noexpand\@@basecount{#1}{#2}{#3}{}}%
    \@baseResult%
  \fi%
}

\newcount\@@baseQ
\newcount\@@baseB
\newcount\@@baseM
\newcount\@@baseR
\newcommand*{\@@basecount}[4]{% Syntax: \@@basecount{alphabet}{alphabet length}{counter}{accumulator}
  \ifnum #3=0%
    #4%
  \else%
    \@@baseQ #3%
    \@@baseB #2%
    \divide\@@baseQ by\@@baseB%
    \@@baseM \@@baseQ%
    \multiply\@@baseM by\@@baseB%
    \@@baseR #3%
    \advance\@@baseR by-\@@baseM%
    \edef\@@baseNext{\the\@@baseQ}%
    \edef\@@baseAcc{\csname @#1\endcsname{\the\@@baseR}#4}%
    \@@basecount{#1}{#2}{\@@baseNext}{\@@baseAcc}%
  \fi%
}

\newcommand*{\baseFive}[1]{\expandafter\@baseFive{\csname c@#1\endcsname}}
\newcommand*{\@baseFive}[1]{\@basecount{}{5}{\number #1}}

\newcommand*{\basicSymb}[1]{\expandafter\@basicSymb{\csname c@#1\endcsname}}
\newcommand*{\@basicSymb}[1]{\@basecount{simpleSymbols}{10}{\number #1}}
\newcommand*{\@simpleSymbols}[1]{$\ifcase#1=\or+\or-\or\times\or\div\or<\or>\or\vee\or\wedge\or\neg\else\@ctrerr\fi$}

\newcommand*{\greekX}[1]{%
  \expandafter\@greekX{\csname c@#1\endcsname}%
}

\newcommand*{\@greekX}[1]{%
  \@basecount{greek}{24}{\number #1}%
}

\newcommand*{\binaryX}[1]{%
  \expandafter\@binaryX\csname c@#1\endcsname%
}

\newcommand*{\@binaryX}[1]{\binary{#1}}

\newcommand*{\octX}[1]{%
  \expandafter\@octX\csname c@#1\endcsname%
}

\newcommand*{\@octX}[1]{0\oct{#1}}

\newcommand*{\hexX}[1]{%
  \expandafter\@hexX\csname c@#1\endcsname%
}

\newcommand*{\@hexX}[1]{0x\hex{#1}}

\newcommand*{\greek}[1]{%
  \expandafter\@greek\csname c@#1\endcsname%
}

\newcommand*{\@greek}[1]{%
  $\ifcase#1%
  \alpha\or\beta\or\gamma\or\delta\or\varepsilon%
  \or\zeta\or\eta\or\theta\or\iota\or\kappa\or\lambda%
  \or\mu\or\nu\or\xi\or o\or\pi\or\rho\or\sigma%
  \or\tau\or\upsilon\or\varphi\or\chi\or\psi\or\omega%
  \or A\or B\or\Gamma\or\Delta\or E%
  \or Z\or H\or\Theta\or I\or K\or\Lambda%
  \or M\or N\or\Xi\or O\or\Pi\or P\or\Sigma%
  \or T\or\Upsilon\or\Phi\or X\or\Psi\or\Omega%
  \else\@ctrerr\fi$%
}


\newcommand*{\symcount}[1]{%
  \expandafter\@symcount\csname c@#1\endcsname%
}

\newcommand*{\@symcount}[1]{%
  $\ifcase#1%
  \neg\or\sqrt{\ }\or \int%
  \or +\or \times\or\pm\or \circ\or \oplus\or \otimes%
  \or\cup \or\cap%\or\sqcup\or\sqcap%
  \or \vert\or \Vert\or \perp%\or\angle%
  \or\sim%
  \or <\or >%
  \or =\or \cong%
  \or \in%
  \or\exists\or\forall\or \vee\or \wedge%
  \or\Rightarrow\or\mapsto\or\Leftrightarrow%
  \or\lightning\or\square%
  \else\@ctrerr\fi$%
}

\newcommand*{\shiftedpage}[1]{% Page counter that can be shifted by Fulbert og Beatrice without disturbing the underlying ``page'' counter. Uses the style stored in \countstyle and the offset in pageoffset.
    \@shiftedpage\csname c@#1\endcsname%
}

\newcommand*{\@shiftedpage}[1]{%
    \@@shiftedpage{\number #1}{\number \c@pageoffset}%
}

\newcount \@shiftedcount
\newcount \@shiftedoffset
\DeclareRobustCommand{\@@shiftedpage}[2]{%
    \@shiftedcount #1%
    \@shiftedoffset #2%
    \advance\@shiftedcount by\@shiftedoffset%
    \csname @\countstyle\endcsname{\@shiftedcount}%
}


\makeatother
